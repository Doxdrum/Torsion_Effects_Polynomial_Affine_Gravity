%%%%%%%%%%%%%%%%%%%%%%% file template.tex %%%%%%%%%%%%%%%%%%%%%%%%%
%
% This is a template file for The European Physical Journal
%
% Copy it to a new file with a new name and use it as the basis
% for your article
%
%%%%%%%%%%%%%%%%%%%%%%%% Springer-Verlag %%%%%%%%%%%%%%%%%%%%%%%%%%
%
\begin{filecontents}{leer.eps}
%!PS-Adobe-2.0 EPSF-2.0
%%CreationDate: Mon Jul 13 16:51:17 1992
%%DocumentFonts: (atend)
%%Pages: 0 1
%%BoundingBox: 72 31 601 342
%%EndComments


%gsave
%72 31 moveto
%72 342 lineto
%601 342 lineto
%601 31 lineto
%72 31 lineto
%showpage
%grestore
%%Trailer
%%DocumentFonts: Helvetica
\end{filecontents}
%
\documentclass[epj]{svjour}
% Remove option referee for final version
%
% Remove any % below to load the required packages
%\usepackage{latexsym}
\usepackage{graphics}
% etc
%

\usepackage[utf8]{inputenc}
\usepackage{amsmath}
\usepackage{amsfonts}
\usepackage{cite} 
\usepackage{breqn}
\usepackage{graphicx}
\usepackage{hyperref}
\allowdisplaybreaks

\providecommand{\Rie}[3]{\mathcal{R}_{#1}{}^{ #2}{}_{#3}}
\providecommand{\ctG}[3]{\Gamma_{#1}{}^{ #2}{}_{#3}}
\providecommand{\ctg}[3]{\gamma_{#1}{}^{ #2}{}_{#3}}
\providecommand{\B}[3]{\mathcal{B}_{#1}{}^{ #2}{}_{#3}}
\providecommand{\P}[3]{\mathcal{P}_{#1}{}^{ #2}{}_{#3}}
\providecommand{\A}[1]{\mathcal{A}_{#1}}
\providecommand{\Ri}[1]{\mathcal{R}_{#1}}
\providecommand{\deV}[1]{\mathrm{d}V^{#1}}


\begin{document}
%
\title{Cosmological solutions with torsion effects in Polynomial Affine Gravity}
%\subtitle{Do you have a subtitle?\\ If so, write it here}
\author{Jos\'e Perdiguero G\'arate \inst{1} \and Oscar Castillo-Felisola \inst{2,3} %\and
        %Jos\'e A. Font \inst{4} 
        \and Bastian Grez \inst{2} \and Gonzalo Olmo \inst{1} 
        \and Oscar Orellana \inst{4} 
%\thanks is optional - remove next line if not needed
%\thanks{\emph{Present address:} Insert the address here if needed}%
}                     % Do not remove
%
%\offprints{}          % Insert a name or remove this line
%
\institute{Departamento de F\'isica Te\'orica and IFIC, Centro Mixto Universidad de
Valencia - CSIC. Universidad de Valencia, Burjassot-46100, Valencia, Spain \and 
Departamento de F\'isica, Universidad T\'ecnica Federico Santa Mar\'ia Casilla 110-V, Valpara\'iso, Chile \and
Centro Cient\'ifico Tecnol\'ogico de Valpara\'iso Casilla 110-V, Valpara\'iso, Chile \and
%Departamento de Astronom\'ia y Astrof\'isica, Universitat de Valencia, Dr. Moliner 50, 46100, Burjassot (Valencia), Spain \and
Departamento de Matem\'aticas, Universidad T\'ecnica Federico Santa Mar\'ia Casilla 110-V, Valpara\'iso, Chile}
%
\date{Received: date / Revised version: date}
% The correct dates will be entered by Springer
%
\abstract{
The Polynomial Affine Gravity it is an alternative gravitational model, where the interactions 
are mediated solely by the affine connection, instead of the metric tensor. In this paper, we
explore the space of solutions to the field equations, when the torsion fields are turned on
inf the frame of cosmology. Moreover, we explore how to generate metric descendant structures 
coming from the space of solutions.
%
\PACS{
      {PACS-key}{discribing text of that key}   \and
      {PACS-key}{discribing text of that key}
     } % end of PACS codes
} %end of abstract
%
\maketitle
%

\section{Introduction}
\label{sec:Introduction}

Einstein's theory of General Relativity (GR) is currently the most successful theory to describe gravitational interactions, exhibiting excellent agreement between theoretical predictions and observational data in a variety of scenarios \cite{Will_2014,will_2018,Weinberg:2008zzc}, from laboratory and solar system scales, to the orbital motions of binary pulsars, cosmology and even gravitational waves from colliding compact objects \cite{Abbott_2016,Abbott_2017} and the shadows of supermassive black holes. 

In its traditional formulation, GR is a theory in which the gravitational interaction is solely mediated by the 
metric tensor \cite{Einstein_GR,Einstein_GR_Bases}, from which other quantities such as a covariant derivative or curvature tensors can be derived. The nonlinear character of the equations make it necessary to use certain strategies and impose symmetries to obtain a simplified form that allows us to perform explicit computations. In cosmological scenarios, for instance, homogeneity and isotropy \cite{Frie,Friedmann,Lema,Lema_Expansion,Robertson_1,Robertson_2,Robertson_3} turn the original set of coupled, nonlinear differential equations for ten independent variables into a second-order equation for a single function, the scale factor  \cite{Einstein_GR_Feqs}. The evolution of this factor is determined by the matter-energy sources, which may include a cosmological constant \cite{Einstein_Cosmological_Constant} or some other form of dark energy, and the curvature of the spatial sections of the foliated space-time. 

Despite its success, GR also faces difficulties that suggest that a more fundamental description of the gravitational interactions is necessary. Its combination with quantum theory indicates that an ultraviolet completion is needed \cite{PhysRev.160.1113,PhysRev.162.1195,PhysRevD.10.401,PhysRevD.10.411}, while the need to include a dark sector, for which no direct evidence exists, that dominates the cosmic evolution and the dynamics of galaxies and clusters \cite{Rotation_Curve,Rotation_Curve_2,NASERI2021100888,10.1093/mnras/stz2757,Riess_1998}, may point towards new gravitational dynamics in the infrared.  As a result, it is becoming generally accepted that GR should be viewed as an effective theory that may require extensions at very short and very large length scales. Obviously, the kind of modifications needed are the million dollar question, and multiple alternatives can be found in the literature. Among those, theories of the $f(R)$ type \cite{10.1093/mnras/150.1.1,STAROBINSKY198099,Hehl_1995,Baldazzi_2022,Vitagliano_2011,Karahan_2012,SARDANASHVILY_2011,OLMO_2011}, scalar-tensor theories, and various extensions of them \cite{ASENS_1924_3_41__1_0,ASENS_1925_3_42__17_0,Kaluza_Klein,Klein,saridakis2023modified,Shankaranarayanan_2022}, among others, have become very popular in the last two decades for theoretical and phenomenological reasons, offering a variety of strategies and alternative mechanisms to justify the accelerated cosmic expansion, inflationary scenarios, and the possible existence of exotic compact objects. 

A fundamental ingredient in the construction of any gravity theory is the type of fields associated with the gravitational interaction. The traditional approach assumes that the underlying geometry is of Riemannian (or pseudo-Riemannian) type, being described solely by the metric tensor. Alternative approaches in which metric and connection are treated as equally fundamental and independent fields are also gaining attention in the last years. The role that torsion and non-metricity could have at cosmic scales and in strong gravity scenarios offers a window to explore new gravitational phenomena beyond the Riemannian framework. This can be viewed as a complementary approach to the modified theories scenario but considering modified geometry instead. 

The freedom contained in the connection, with up to 64 independent components, offers a vast range of options to generate new gravitational phenomena and even to potentially accommodate adaptations needed to build a framework more suitable to incorporate quantum phenomena \cite{Hehl:1994ue}. In this sense, it is important to note that the most successful description of the fundamental interactions is based on gauge theories, which encode the dynamics of the connections of the symmetry groups of the standard model of elementary particles. It is thus natural to consider if a purely connection-based formulation of gravity is possible, such that it could be represented in a form more closely related to the other interactions. The solution to this question is far from trivial, though some examples of purely affine theories exist in the literature.  

Looking back at the literature on purely affine gravity theories, the model proposed by  Sir Arthur  Eddington \cite{Eddington1923-EDDTMT,schrodinger1985space} represents a simple example that nonetheless illustrates the key challenges faced by this type of theories. Eddington's theory is defined by the square root of the determinant of the symmetric part of the Ricci tensor, which is a diffeomorphism invariant quantity \cite{eisenhart1972non}. The variation of this action (with respect to the affine connection) can be manipulated to obtain the well-known Einstein equations in vacuum coupled to an arbitrary cosmological constant (see, for instance, \cite{POP_AWSKI_2007}, where the role of the antisymmetric part of the Ricci is also analyzed). As a result, the Ricci tensor can be interpreted as an emergent metric tensor (as long as the cosmological constant does not vanish). Despite is formal resemblance with GR, Eddington's theory faces evident difficulties when trying to couple gravity to matter fields, as there is no clear mechanism to build a suitable matter action in the absence of a metric tensor. Some recent attempts in this direction can be found in \cite{Knorr_2021,POP_AWSKI_2008,Pop_awski_2009,Filippov_2010,Azri_2015}, where  a metric tensor, that couples only to the matter sector is considered. Other formulations of purely affine theories have been inspired by the canonical approach to quantum gravity \cite{Krasnov_2011}, where the only dynamical field is an $SU(2)$ connection. Eddington's theory has also inspired metric-affine formulations  in recent years in the form of determinantal Born-Infeld like actions \cite{BI_Gravity,Deser_1998,Vollick:2003qp,Banados:2010ix,Jim_nez_2021,Afonso:2021aho}. 

An alternative approach to the purely affine formulation of gravity is provided by the Polynomial Affine Gravity (PAG) model. The PAG action is designed following a reasoning that parallels the \textit{dimensional analysis} technique of field theories, considering the irreducible terms that can be constructed out of the affine connection and its first derivatives and that preserve the invariance under diffeomorphisms. Because of the absence of a metric tensor, there is a geometric constraint to satisfy in order to build the most general scalar densities in the affine geometry, which leads to a finite number of terms in the action. This property is usually referred to as the \textit{rigidity} of the model.

Moreover, it is possible to couple a scalar field to the affine model using the same \textit{dimensional analysis} principle and, as a consequence, the \textit{rigidity} of the model is inherited by the coupling mechanism. This approach avoids the use of a metric to build the matter action, bypassing also the difficulties of Eddington's approach, and provides a new landscape to build purely affine gravity theories. The model has been studied in Refs.\cite{castillofelisola2016polynomial,castillofelisola2016einsteins,castillofelisola2019cosmological,Castillo_Felisola_2018,Castillo_Felisola_2020,Castillo_Felisola_2022_EPJC,Castillo_Felisola_2022_Universe} and in this work, we explore the consequences of including torsion effects coming from the antisymmetric part of the affine connection in four dimensions in cosmological settings.

The paper is organized as follows: In Section \ref{sec:PAG}, we present a brief overview on how to built the polynomial affine model of gravity, highlighting 
its features and the method to build up the Ansatze compatible with the cosmological symmetries. A complete scan of cosmological solutions to the field equations is presented in Section \ref{sec:solutions}. In Section \ref{sec:analysis}, we analyses and discuss the cosmological solutions, and provided a physical interpretation of the solutions by obtaining metric-descendant structures coming
from the irreducible fields of the affine connection. Final remarks are presented in Section \ref{sec:final_remarks}. 

\section{Polynomial Affine Gravity}
\label{sec:PAG}

As said earlier, the Polynomial Affine Gravity is an alternative gravitational model whose fundamental field is  the affine connection, endowing the manifold only with an affine structure $\mathcal{M}(\Gamma)$. In order to build the action, it is convenient to decomposes the affine connection as
\begin{equation}
\begin{aligned}
    \label{affine_connection}
    \hat{\Gamma}_{\alpha}{}^{\beta}{}_{\gamma} & = \hat{\Gamma}_{(\alpha}{}^{\beta}{}_{\gamma)} +  \hat{\Gamma}_{[\alpha}{}^{\beta}{}_{\gamma]},  \\
    & = \ctG{\alpha}{\beta}{\gamma} + \B{\alpha}{\beta}{\gamma} + \delta^{\beta}_{[\gamma}\A{\alpha]},
\end{aligned}
\end{equation}
where the first term corresponds to the symmetric part of the connection $\hat{\Gamma}_{(\alpha}{}^{\beta}{}_{\gamma)} = \ctG{\alpha}{\beta}{\gamma}$, and
the last two terms are related to the torsion tensor. The former represents  the purely tensorial (traceless) part of the torsion $\B{\alpha}{\beta}{\gamma}$, while the latter is a pure vectorial object $\A{\alpha}$.
Additionally,  the introduction of the volume element is necessary and, in the absence of a metric tensor, one can use the wedge product to define it as  $\deV{\alpha\beta\gamma\delta} = \mathrm{d}x^{\alpha}\wedge\mathrm{d}x^{\beta}\wedge
\mathrm{d}x^{\gamma}\wedge\mathrm{d}x^{\delta}$. It is worth emphasizing that the action must preserve the invariance under diffeomorphisms, which is broken by the symmetric part of the affine connection, and as a consequence of this,
the symmetric part must appear in the action only through the covariant derivative $\ctG{\alpha}{\beta}{\gamma} \to \nabla^\Gamma$.
Therefore, the fundamental building blocks of the affine model are 
\begin{equation}
\label{irreducible_fields}
\nabla^\Gamma_\alpha, \B{\alpha}{\beta}{\gamma}, \mathcal{A}_\alpha, \mathrm{d}V^{\alpha\beta\gamma\delta}.
\end{equation}

In order to build up the action, we use a sort of \textit{dimensional analysis} technique which has been reviewed in \cite{castillofelisola2016einsteins,Castillo_Felisola_2020}.
The method allows one to consider every scalar densities composed by powers of Eq. \eqref{irreducible_fields} by using the operators $\mathcal{N}$ and $\mathcal{W}$ which count the number of free indices and the weight of the field, respectively. The analysis provides a geometrical constraint equation that limits the number of configurations. For each configuration, there are multiple permutations that need to be analyzed by taking into account the symmetries of the fundamental fields ( see \cite{castillofelisola2016einsteins,Castillo_Felisola_2020}  for more details on this procedure). 
A three-dimensional version of this model was developed following this approach in Refs. \cite{Castillo_Felisola_2022_EPJC,Castillo_Felisola_2022_Universe}.

The most general action (up to topological invariants and boundary terms) in four dimensions is given by
\begin{equation}
\label{PAG_action}
\begin{split}
S & = \int  \mathrm{d}V^{\alpha \beta \gamma \delta} \bigg[
      B_1 \mathcal{R}_{\mu\nu}{}^{\mu}{}_{\rho}\mathcal{B}_{\alpha}{}^{\nu}{}_{\beta}\mathcal{B}_{\gamma}{}^{\rho}{}_{\delta}
    + B_2 \mathcal{R}_{\alpha\beta}{}^{\mu}{}_{\rho} \mathcal{B}_{\gamma}{}^{\nu}{}_{\delta} \mathcal{B}_{\mu}{}^{\rho}{}_{\nu}
    \\
    & \quad
    + B_3 \mathcal{R}_{\mu\nu}{}^{\mu}{}_{\alpha} \mathcal{B}_{\beta}{}^{\nu}{}_{\gamma} \mathcal{A}_\delta
    + B_4 \mathcal{R}_{\alpha\beta}{}^{\sigma}{}_{\rho}\mathcal{B}_{\gamma}{}^{\rho}{}_{\delta}\mathcal{A}_\sigma
    \\
    & \quad
    + B_5 \mathcal{R}_{\alpha \beta}{}^{\rho}{}_{\rho} \mathcal{B}_{\gamma}{}^{\sigma}{}_{\delta} \mathcal{A}_\sigma
    + C_1 \mathcal{R}_{\mu\alpha}{}^{\mu}{}_{\nu} \nabla_\beta \mathcal{B}_{\gamma}{}^{\nu}{}_{\delta}
    \\
    & \quad
    + C_2 \mathcal{R}_{\alpha\beta}{}^{\rho}{}_{\rho} \nabla_\sigma \mathcal{B}_{\gamma}{}^{\sigma}{}_{\delta}
    + D_1 \mathcal{B}_{\nu}{}^{\mu}{}_{\lambda} \mathcal{B}_{\mu}{}^{\nu}{}_{\alpha} \nabla_\beta \mathcal{B}_{\gamma}{}^{\lambda}{}_{\delta}
    \\
    & \quad
    + D_2 \mathcal{B}_{\alpha}{}^{\mu}{}_{\beta} \mathcal{B}_{\mu}{}^{\lambda}{}_{\nu} \nabla_{\lambda} \mathcal{B}_{\gamma}{}^{\nu}{}_{\delta}
    + D_3 \mathcal{B}_{\alpha}{}^{\mu}{}_{\nu}\mathcal{B}_{\beta}{}^{\lambda}{}_{\gamma} \nabla_\lambda \mathcal{B}_{\mu}{}^{\nu}{}_{\delta}
    \\
    & \quad
    + D_4 \mathcal{B}_{\alpha}{}^{\lambda}{}_{\beta}\mathcal{B}_{\gamma}{}^{\sigma}{}_{\delta}\nabla_\lambda \mathcal{A}_\sigma
    + D_5 \mathcal{B}_{\alpha}{}^{\lambda}{}_{\beta} \mathcal{A}_\sigma \nabla_\lambda \mathcal{B}_{\gamma}{}^{\sigma}{}_{\delta}
    \\
    &\quad
    + D_6 \mathcal{B}_{\alpha}{}^{\lambda}{}_{\beta}\mathcal{A}_\gamma \nabla_\lambda A_\delta
    + D_7\mathcal{B}_{\alpha}{}^{\lambda}{}_{\beta} \mathcal{A}_\lambda \nabla_\gamma A_\delta
    \\
    & \quad
    + E_1\nabla_\rho \mathcal{B}_{\alpha}{}^{\rho}{}_{\beta} \nabla_\sigma \mathcal{B}_{\gamma}{}^{\sigma}{}_{\delta}
    + E_2 \nabla_\rho \mathcal{B}_{\alpha}{}^{\rho}{}_{\beta} \nabla_\gamma \mathcal{A}_\delta
    \\
    & \quad
    + F_1 \mathcal{B}_{\alpha}{}^{\mu}{}_{\beta} \mathcal{B}_{\gamma}{}^{\sigma}{}_{\delta} \mathcal{B}_{\mu}{}^{\lambda}{}_{\rho} \mathcal{B}_{\sigma}{}^{\rho}{}_{\lambda}
    + F_2\mathcal{B}_{\alpha}{}^{\mu}{}_{\beta} \mathcal{B}_{\gamma}{}^{\nu}{}_{\lambda} \mathcal{B}_{\delta}{}^{\lambda}{}_{\rho} \mathcal{B}_{\mu}{}^{\rho}{}_{\nu}
    \\
    &\quad
    + F_3 \mathcal{B}_{\nu}{}^{\mu}{}_{\lambda} \mathcal{B}_{\mu}{}^{\nu}{}_{\alpha} \mathcal{B}_{\beta}{}^{\lambda}{}_{\gamma} \mathcal{A}_\delta
    + F_4 \mathcal{B}_{\alpha}{}^{\mu}{}_{\beta}\mathcal{B}_{\gamma}{}^{\nu}{}_{\delta}\mathcal{A}_\mu \mathcal{A}_\nu \bigg].
    \end{split}
\end{equation}
In the above action, the covariant derivative and the curvature tensor are defined
with respect to the symmetric part of the connection, meaning that $\nabla = \nabla^{\Gamma}$ 
and $\mathcal{R} = \mathcal{R}^{\Gamma}$. 

One of the most important features of this model is that the lack of a metric tensor  implies that the number of terms in the action is finite. This property is known as the \textit{rigidity} of the model.
The fact that all the coupling constants are dimensionless  suggests that the model is power-counting renormalizable, so that in the hypothetical scenario of its quantization all possible counter-terms should have the form of the ones already written in Eq. \eqref{PAG_action}.  This is something desirable from a stand point of Quantum Field Theory view. The reason is that the superficial degree of divergence vanishes. Additionally, the dimensionless nature of the coupling constants also suggests a conformal symmetry, at least at a classical level. An explicit implementation of this idea is likely to require an understanding of the projective invariance of the theory, along the lines of \cite{Olmo:2022ops}, though this aspect will not be explored in this paper. 

In the torsion-free sector the field equations coming from varying the action is a generalization
of the Einstein's vacuum field equations. In that sense, the space of solutions of general relativity
without an energy-momentum tensor, is a subspace of solutions of the Polynomial Affine Gravity.
Finally, it is possible to couple a scalar field using the \textit{dimensional analysis} technique introduced above without the necessity of having a metric structure on the manifold. The kinetic terms would couple to a combination of tensors and tensor densities, while the potential term should rescale the volume element. ln the torsionless limit, the resulting equations turn out to recover Einsteins's theory coupled to a scalar field, see Ref.\cite{castillofelisola2023inflationary}, giving support to this approach.

Since we are interested in the study of the cosmology, we want to impose the symmetries of the cosmological principle on 
the irreducible fields coming from the affine connection, which are $\Gamma$, $\mathcal{B}$ and $\mathcal{A}$. 
In order to build up the ansatz we compute the Lie derivative of each irreducible field along the Killing vectors $\xi_i$ 
that generates the symmetries of homogeneity (translations) $\mathcal{P}_i$ and isotropy (rotations) $\mathcal{J}_i$. 
A derivation of the Killing vectors along with an explicit computation of the Lie derivative can be found 
in Ref. \cite{Castillo-Felisola17}. In what follows we shall show only the results.

The computation of the Lie derivative of $\Gamma$ along the Killing vectors, determined its coefficients: 
\begin{align}
    \label{G_ansatz}
    \ctG{t}{t}{t} & =f(t), & \ctG{i}{t}{j} & = g(t) S_{i j}, \\
    \ctG{t}{i}{j} &= h(t), \delta^{i}_{j} & \ctG{i}{j}{k} & = \ctg{i}{j}{k},
\end{align}
where $S_{ij}$ is the three-dimensional rank two symmetric tensor defined as follow:
\begin{equation*}
    S_{i j}=\left(\begin{array}{ccc}
    \frac{1}{1-\kappa r^2} & 0 & 0 \\
    0 & r^2 & 0 \\
    0 & 0 & r^2 \sin ^2 \theta
    \end{array}\right),
\end{equation*}
and $\gamma$ is the three-dimensional symmetric connection compatible with desired symmetries, written as:
\begin{align*}
    \ctg{r}{r}{r} & = \frac{\kappa r}{1 - \kappa r^2}, & \ctg{\theta}{r}{\theta} & = \kappa r^3 - r, \\
    \ctg{\varphi}{r}{\varphi} & = \left(\kappa r^3 - r\right)\sin^2\theta, & \ctg{r}{\theta}{\theta} & = \frac{1}{r}, \\
    \ctg{\varphi}{\theta}{\varphi} & = -\cos\theta\sin\theta, & \ctg{r}{\varphi}{\varphi} & = \frac{1}{r}, \\
    \ctg{\theta}{\varphi}{\varphi} & = \frac{\cos \theta}{\sin \theta}.
\end{align*}
Additionally, the affine function $f(t)$ can be set to zero by a re-parametrization of the $t$ coordinate \cite{Castillo_Felisola_2022_EPJC}. 
Therefore, there are only two nontrivial functions to define completely the symmetric part of the connection.

Following a similar procedure for the torsion tensor. It is possible to determined the ansatz compatible with 
the required symmetries. For its traceless part $\mathcal{B}$, the non-vanishing components are:
\begin{equation}
\label{B_ansatz}
\begin{aligned}
    \B{\theta}{r}{\varphi} & = \psi (t) r^2\sin\theta \sqrt{1 - \kappa r^2}, &
    \B{r}{\theta}{\varphi} & =\frac{\psi (t) \sin \theta}{\sqrt{1 - \kappa r^2}}, \\
    \B{r}{\varphi}{\theta} & =\frac{\psi(t)}{ \sqrt{1-\kappa r^{2}} \sin \theta},
\end{aligned}
\end{equation}
while the nontrivial components of its vectorial part $\mathcal{A}$ is:
\begin{equation}
    \label{A_ansatz}
    \A{t} = \eta(t).
\end{equation}

Finally, the field equations are obtained by varying the action with respect to each irreducible field using Kijowkski's 
formalism, see Ref. \cite{KJ_Formalism,Castillo_Felisola_2020}. The complete  set of the field equation for 
each irreducible field can be found in Refs.\cite{Castillo_Felisola_2020,Castillo-Felisola_2023}.

\section{Cosmological solutions}
\label{sec:solutions}

Using the ansatz in Eqs. \eqref{G_ansatz}, \eqref{B_ansatz} and \eqref{A_ansatz} the field equation are
\begin{dmath}
    \label{Feq_1}
    \left(B_3\left(\dot{g} + gh + 2\kappa\right) - 2B_4\left(\dot{g} - gh\right) + 2D_6\eta g - 2F_3\psi^2\right)\psi = 0,
\end{dmath}
\begin{dmath}
    \label{Feq_2}
    \left(B_3\eta\psi -2B_4\eta\psi + C_1\left(\dot{\psi} - 2h\psi\right)\right)g = 0,
\end{dmath}
\begin{dmath}
    \label{Feq_3}
    \left(B_3 + 2B_4\right)\eta g\psi + 2C_1\left(\kappa\psi + 4gh\psi - g\dot{\psi} - \psi\dot{g}\right) + 2\psi^3\left(2D_2 - D_1 - D_3\right) = 0,
\end{dmath}
\begin{dmath}
    \label{Feq_4}
    B_3\left(\eta\left(h\psi - \dot{\psi}\right) -\psi\dot{\eta}\right) - 2B_4\left(\eta\left(-h\psi - \dot{\psi}\right) -\psi\dot{\eta}\right) 
    + C_1\left(4h^2\psi + 2\psi\dot{h} -\ddot{\psi}\right) + D_6\eta^2\psi = 0,
\end{dmath}
\begin{dmath}
    \label{Feq_5}
    B_3\left(\dot{g} + gh + 2\kappa\right)\eta - 2B_4\left(\dot{g} - gh\right)\eta + C_1\left(2\kappa h + 4gh^2 + 2g\dot{h} - \ddot{g}\right) +
    6h\psi^2\left(2D_2 - D_1 - D_3\right) + D_6 \eta^2 g - 6F_3\eta\psi^2 = 0
\end{dmath}
where $\mathcal{A}$ and $\mathcal{B}$ yield one equation each Eq. \eqref{Feq_1} and Eq. \eqref{Feq_5} respectively, 
while the remaining three come from $\Gamma$. Notice, we have four unknown functions 
of time $g(t)$, $h(t)$, $\psi(t)$ and $\eta(t)$ while there are five differential equations meaning the system is overdetermined.

Nonetheless, we will show that it can actually be solved analytically without any assumption. For this purpose,
we have developed a \textit{logical scheme} that allow us to seek systematically the solutions by branches. First, notice Eqs. \eqref{Feq_1} 
and \eqref{Feq_2} follow the form
\begin{dmath}
    \label{Feq_1_1}
    \mathcal{F}(g, \dot{g}, h,\psi,\eta)\psi = 0,
\end{dmath}
\begin{dmath}
    \label{Feq_2_2}
    \mathcal{G}(h,\psi, \dot{\psi}, \eta)g = 0,
\end{dmath}
where
\begin{dmath}
\mathcal{F}(g,\dot{g},h,\psi,\eta)  = B_3\left(\dot{g} + gh + 2\kappa\right) - 2B_4\left(\dot{g} - gh\right) + 2D_6\eta g - 2F_3\psi^2.
\end{dmath}
\begin{dmath}
\mathcal{G}(h,\psi, \dot{\psi}, \eta) = B_3\eta\psi -2B_4\eta\psi + C_1\left(\dot{\psi} - 2h\psi\right).
\end{dmath}
Thus, using Eqs. \eqref{Feq_1_1} and \eqref{Feq_2_2} it is possible to distinguish four different branches:
\begin{align}
    \textbf{First branch:}  &&& \mathcal{F}(g,h,\psi,\eta) \! = 0 \! \wedge \! \mathcal{G}(h,\psi,\eta) \!= \! 0,  \label{branch_1}\\
    \textbf{Second branch:} &&& \mathcal{F}(g,h,\psi,\eta)  = 0  \wedge  g  = 0, \label{branch_2}\\
    \textbf{Third branch:}  &&& \mathcal{G}(h,\psi,\eta)   = 0   \wedge  \psi  = 0, \label{branch_3}\\
    \textbf{Fourth branch:} &&& \psi  = 0  \wedge  g  = 0. \label{branch_4}
\end{align}
In terms of the richness of the solutions, the most interesting branch is that from Eq. \eqref{branch_1},
which is the most general case, while the simplest is the one characterized by Eq. \eqref{branch_4}.

\subsection{First branch}
\label{sec:first_branch}

This branch contains the most general case where the field equations are 
\begin{dmath}
    \label{Feq_B1_1}
    B_3\left(\dot{g} + gh + 2\kappa\right) - 2B_4\left(\dot{g} - gh\right) + 2D_6\eta g - 2F_3\psi^2 = 0,
\end{dmath}
\begin{dmath}
    \label{Feq_B1_2}
    B_3\eta\psi -2B_4\eta\psi + C_1\left(\dot{\psi} - 2h\psi\right) = 0,
\end{dmath}
\begin{dmath}
    \label{Feq_B1_3}
    \left(B_3 + 2B_4\right)\eta g\psi + 2C_1\left(\kappa\psi + 4gh\psi - g\dot{\psi} - \psi\dot{g}\right) + 2\psi^3\left(2D_2 - D_1 - D_3\right) = 0,
\end{dmath}
\begin{dmath}
    \label{Feq_B1_4}
    B_3\left(\eta\left(h\psi - \dot{\psi}\right) -\psi\dot{\eta}\right) - 2B_4\left(\eta\left(-h\psi - \dot{\psi}\right) -\psi\dot{\eta}\right) 
    + C_1\left(4h^2\psi + 2\psi\dot{h} -\ddot{\psi}\right) + D_6\eta^2\psi = 0,
\end{dmath}
\begin{dmath}
    \label{Feq_B1_5}
    B_3\left(\dot{g} + gh + 2\kappa\right)\eta - 2B_4\left(\dot{g} - gh\right)\eta + C_1\left(2\kappa h + 4gh^2 + 2g\dot{h} - \ddot{g}\right) +
    6h\psi^2\left(2D_2 - D_1 - D_3\right) + D_6 \eta^2 g - 6F_3\eta\psi^2 = 0.
\end{dmath}

We solve Eq.\eqref{Feq_B1_2} to find an expression for $\eta(t)$
\begin{equation}
    \label{B1_eta}
    \eta(t) = \left(\frac{2h\psi - \dot{\psi}}{\psi}\right)\left(\frac{C_1}{B_3 - 2B_4}\right).
\end{equation}
Replacing the above expression for $\eta(t)$ into Eq. \eqref{Feq_B1_4}, leads to two sub-branches for the $h(t)$ function either
\begin{align}
    \label{B1_h}
    h(t) & = \frac{\dot{\psi}}{2\psi} & & \wedge &  h(t) & = \frac{\dot{\psi}}{\psi}\left(\frac{C_1 D_6}{3B_3^2 - 8B_3B_4 + B_4^2 + 2C_1D_6}\right).
\end{align}
Using the simplest form of $h(t)$ function, then Eq. \eqref{Feq_B1_3} turns to
\begin{equation}
    -2\left(D_1 - 2D_2 + D_3\right)\psi^3 + 2C_1\left(\psi\left(\kappa - \dot{g}\right) + g\dot{\psi}\right) = 0,
\end{equation}
which allows to solve $g(t)$ in terms of the $\psi(t)$ 
\begin{equation}
    \label{B1_g}
    g(t) = \psi(t) \left(g_0 + \int_1^t \left(\frac{\kappa}{\psi(\tau)} - \psi(\tau) \left(\frac{D_1 - 2D_2 + D_3}{C_1}\right)\right) \mathrm{d}\tau\right),
\end{equation}
where $g_0$ is an integration constant. The above expression also solve Eq. \eqref{Feq_B1_5}. Then, Eq. \eqref{Feq_B1_1} becomes 
an integro-differential equation of first order
\begin{dmath}
    \label{psi_integro_diff_equation}
    \dot{\psi}\left(g_0 + \int_1^t \left(\frac{\kappa}{\psi(\tau)} - \psi(\tau) \alpha\right) \mathrm{d}\tau\right)\beta -
    \psi^2 \gamma + 2\kappa\beta = 0.
\end{dmath}
where $\alpha$, $\beta$ and $\gamma$ are related to the coupling constants
by the following relation
\begin{align}
    \alpha & = \left(\frac{D_1 - 2D_2 + D_3}{C_1}\right), \\
    \beta & = \left(\frac{3B_3 - 2B_4}{2}\right), \\
    \gamma & = \left(\beta - 2B_3\right)\alpha + 2F_3,
\end{align}
Then eq.\eqref{psi_integro_diff_equation} can be solved for the special case $\kappa = 0$, with the variable change $\psi (t) = \dot{\phi}(t)$, 
then Eq. \eqref{psi_integro_diff_equation} turns into a second order differential equation for $\phi(t)$
\begin{dmath}
    \ddot{\phi}\left(g_0  - \phi\alpha\right)\beta - \dot{\phi}^2 \gamma  = 0,
\end{dmath}
whose solution is
\begin{equation}
    \phi(t) = \frac{g_0}{\alpha} + \frac{\left(\phi_0 \left(\alpha\beta + \gamma\right) \left(t +  \phi_1\right)\right)^{\frac{\alpha\beta}{\alpha\beta + \gamma}}}{\alpha\beta},
\end{equation}
where $\phi_0$ and $\phi_1$ are integration constant. Using the above solution, it is straightforward to recover the original function
\begin{equation}
    \psi(t) = \phi_0 \left(\left(\alpha\beta + \gamma\right)\phi_0\left(t + \phi_1\right)\right)^{-\frac{\gamma}{\alpha\beta + \gamma}}.
\end{equation}
Using the above expression and from the relations in Eqs. \eqref{B1_eta}, \eqref{B1_h} and \eqref{B1_g} it is direct to find the rest of the
affine functions
\begin{dmath}
\eta(t)  = 0
\end{dmath}
\begin{dmath}
h(t) = - \frac{\gamma}{2\left(\alpha\beta + \gamma\right)\left(t + \phi_1\right)}
\end{dmath}
\begin{multline}
g(t)  = \phi_0 \left(\left(\alpha\beta + \gamma\right)\phi_0 \left(t + \phi_1\right)\right)^{-\frac{\gamma}{\alpha\beta + \gamma}} \\
\left(g_0 - \frac{\left(\left(\alpha\beta + \gamma\right)\phi_0\left(t + \phi_1\right)\right)^{\frac{\alpha\beta}{\alpha\beta + \gamma}}}{\beta}\right)
\end{multline}


\subsection{Second branch}

The second branch imposes the $g(t) = 0$ restriction which leads to a reduced field equations
\begin{dmath}
    \label{Feq_B2_1}
    2\kappa B_3 \psi - 2F_3\psi^3= 0,
\end{dmath}
%\begin{dmath}
%    \label{Feq_B2_2}
%    g = 0,
%\end{dmath}
\begin{dmath}
    \label{Feq_B2_3}
    2\kappa C_1\psi - 2\psi^3 \left(D_1 - 2D_2 + D_3\right) = 0
\end{dmath}
\begin{dmath}
    \label{Feq_B2_4}
    \kappa\left(4h^2C_1 + h\eta\left(B_3 + 2B_4\right) + D_6\eta^2 + 2C_1\dot{h} - \dot{\eta}\left(B_3 - 2B_4\right)\right) = 0,
\end{dmath}
\begin{dmath}
    \label{Feq_B2_5}
    2h \left(\kappa C_1 - 3\left(D_1 - 2D_2 + D_3\right) \psi^2\right) + 2\eta\left(\kappa B_3 - 3F_3\psi^2\right) = 0.
\end{dmath}
Considering Eq. \eqref{Feq_B2_1} its possible to find an expression for $\psi(t)$
\begin{equation}
    \psi(t) = \pm \sqrt{\frac{\kappa B_3}{F_3}},
\end{equation}
then, replacing the above algebraic expression in Eq.\eqref{Feq_B2_5} leads to a relation between $h(t)$ and $\eta(t)$ as follow
\begin{equation}
    h(t) = - \frac{2B_3F_3 \eta}{3B_3 \left(D_1 - 2D_2 + D_3\right) - C_1F_3}.
\end{equation}
Because of the above relation between the functions, Eqs. \eqref{Feq_B2_3} becomes a constraint equation for the coupling constant,
which is solved for the coupling constant $D_1$ as follow\footnote{This solution is only valid for $\kappa \neq 0$.}
\begin{equation}
    \label{B2_k_constraint_1}
	 D_1 = \frac{2B_3D_2 - B_3D_3 + C_1F_3}{B_3}.
\end{equation} 
The constraint for the coupling constant turns Eq. \eqref{Feq_B2_4} into a first order differential equation for $\eta(t)$
\begin{equation}
    \label{B2_k_constraint_2}
    \dot{\eta}C_1 \kappa \left(2B_4 - 3B_3\right) + \eta^2 \left(3B_3^2 - 2B_3 B_4 + C_1D_6\right)\kappa = 0,
\end{equation}
which can be solved by
\begin{equation}
    \eta(t) = \frac{C_1\left(2B_4 - 3B_3\right)}{\left(3B_3 - 2B_4\right)\left(tB_3 + \eta_0 C_1\right) + tC_1D_6},
\end{equation}
then, the other the $h(t)$ can be written as follow
\begin{equation}
    h(t) = \frac{B_3\left(3B_3 - 2B_4\right)}{\left(3B_3 - 2B_4\right)\left(tB_3 + \eta_0 C_1\right) + tC_1D_6} 
\end{equation}
It is important to notice that the above solution is valid only for the special case where $\kappa \neq 0$.
This can bee seen directly from Eqs. \eqref{Feq_B2_1}, \eqref{B2_k_constraint_1} and \eqref{B2_k_constraint_2}. 

If we impose the constraint $\kappa = 0$, then Eq. \eqref{Feq_B2_1} tells us that $\psi(t) = 0$ solves entirely the other
equations. Therefore, the other functions remain unknown.

\subsection{Third branch}

The constraint $\psi(t)=0$ imposes a strong restriction to the systems of differential equations, by reducing into
\begin{dmath}
    g\eta^2D_6 + 2B_4\eta \left(gh - \dot{g}\right) + B_3\eta\left(2\kappa + gh + \dot{g}\right) + 
    C_1\left(2h\left(\kappa + 2gh\right) + 2g\dot{h} - \ddot{g}\right) = 0.
\end{dmath}
The above differential equation has three unknown functions of time $h(t)$, $g(t)$ and $\eta(t)$ which cannot be solved without 
further restriction, or by providing an ansatz for two functions.

\subsection{Fourth branch}

In the fourth branch defined by $g(t) = 0$ and $\psi (t) = 0$, the field equations are reduced
to a single algebraic equation
\begin{equation}
    \kappa\left(hC_1 + B_3\eta\right) = 0,
\end{equation}
the above equation can be solved by setting $\kappa = 0$, then, the functions $h(t)$ and $\eta(t)$ are undetermined,
or by the relation $\eta(h) = - h(t)\left(\frac{C_1}{B_3}\right)$, where $h(t)$ is an arbitrary function.

\section{Analysis of the solutions}
\label{sec:analysis}

The only nontrivial solutions can be found in Sec. \ref{sec:first_branch}, thus, in the following we shall
focus our analysis in that branch. In order to simplify our calculations, and without any lost of generality, we shall
set the integrations constant to $\phi_0 \to 1$, $\phi_1 \to 0$ and $g_0 \to 0$. Hence, the affine functions 
are simplify up to
\begin{align}
	h(t) & = \frac{\gamma}{2t\left(\alpha\beta + \gamma\right)}, \label{sol_h}\\
	g(t) & = -\alpha^2\beta \left(t\left(\alpha\beta + \gamma\right)\right)^{\frac{2\alpha\beta}{\alpha\beta + \gamma} -1}, \label{sol_g} \\
	\psi(t) & = \alpha\beta\left(t\left(\alpha\beta + \gamma\right)\right)^{-\frac{\gamma}{\alpha\beta + \gamma}},  \label{sol_p}\\ 
	\eta(t) & = 0, \label{sol_n}
\end{align}
where $h(t)$ and $g(t)$ define the symmetric part of $\hat{\Gamma}$, whereas
$\psi(t)$ and $\eta(t)$ define the anti symmetric part.

\subsection{Emergent metrics}

Up to this point we have found so far cosmological solutions with non trivial torsion
effects, allowing us to define completely the affine connection, and, consequently
the notion of parallelism. However, the physical implications of the solutions can be
found through the descendant metric structures coming from the irreducible fields 
of the affine connection in the space of solutions to the field equations. 
For starts, lets recall the definition of a metric tensor

\textbf{Definition}. Let $\mathcal{M}$ be a smooth manifold of dimension $n$. At each
point $p \in \mathcal{M}$ there is a vector space $T_p\mathcal{M}$, called the tangent
space. A metric tensor at the point $p$ is a function $g_p (X_p, Y_p)$ which takes as 
inputs a pair of tangent vectors $X_p$ and $Y_p$ at $p$ and produces a real number, 
so that the following conditions are satisfied: i) $g$ is bilinear, meaning that it is
linear sperately in each argument, ii) $g$ is symmetric provided that for all vector $X_p$ and
$Y_p$ we have $g_p(X_p, Y_p) = g_p (Y_p, X_p)$, iii) $g$ is nondegenerate, and therefore the
tensor can be inverted.

In the space of solution, the first emergent metric comes from the Ricci tensor $\mathcal{R}_{\mu\nu}$, which is 
defined by the contraction of the Riemann tensor.\footnote{The Riemann tensor is defined 
by the commutator of covariant derivatives acting on a vector, therefore, it does not 
required the existence of a metric structure.} 
\begin{equation}
    \label{chain}
    \hat{\Gamma} \to \nabla^{\Gamma} \to \Rie{\alpha\beta}{\gamma}{\delta} \to \Ri{\beta\delta}.
\end{equation}
A second metric structure, comes from the contraction of the product of two torsion 
tensor.\footnote{This comes from the antisymmetric part of the affine connection} 
This idea was first introduce by Poplwaski in Ref. \cite{Pop_awski_2013}, and the metric 
structure is defined as follow
\begin{equation}
    \mathcal{P}_{\alpha\delta} = \left(\B{\alpha}{\beta}{\gamma} + \delta^{\beta}_{[\gamma}\A{\alpha]}\right)\left(\B{\beta}{\gamma}{\delta} + \delta^{\gamma}_{[\delta}\A{\beta]}\right).
\end{equation}

Using the cosmological ansatz presented in Eq. \eqref{G_ansatz}, the Ricci tensor is 
defined as follow
\begin{align}
    \label{Ricci_tensor}
    \mathcal{R}_{tt} & = \dot{h} + h^2, & \mathcal{R}_{rr} & = \frac{\dot{g} + gh + 2\kappa}{1 - \kappa r^2}.
\end{align}
The Poplawski's metric is computed using Eqs. \eqref{B_ansatz} and \eqref{A_ansatz}
\begin{align}
    \label{Pop_tensor}
    \mathcal{P}_{tt} & = \eta^2, & \mathcal{P}_{rr} & = -\frac{2\psi^2}{1 - \kappa r^2}.
\end{align}
In order to use the Ricci tensor $\mathcal{R}_{\mu\nu}$ or the Poplawski $\mathcal{P}_{\mu\nu}$ as an emergent metric tensor, is necessary that the tensor 
is well defined, meaning that it must be invertible. If the
tensors are well behaved, then the factors $\left(\dot{g} + gh + 2\kappa\right)$ or $\left(-2\psi^2\right)$ 
could play an analogue role to the scale factor $a^2(t)$ from FLRW.

Using Eqs. \eqref{sol_h}, \eqref{sol_g} and \eqref{Ricci_tensor} we compute the components 
of the Ricci tensor
\begin{align}
    \mathcal{R}_{tt} & = -\frac{3\gamma\left(2\alpha\beta + 3\gamma\right)}{4t^2\left(\alpha\beta + \gamma\right)^2}, \label{R_temporal} \\
    \mathcal{R}_{rr} & = \Sigma\left(\alpha,\beta,\gamma\right)t^{2\left(\frac{-\gamma}{\alpha\beta + \gamma}\right)} \label{R_spatial},
\end{align}
where the constant $\Sigma\left(\alpha,\beta,\gamma\right)$ is defined as
\begin{equation}
    \Sigma\left(\alpha,\beta,\gamma\right) = -\frac{\alpha^2\beta\left(2\alpha\beta - 3\gamma\right)\left(\alpha\beta + \gamma\right)^{\frac{2\alpha\beta}{\alpha\beta + \gamma}}}{2\left(\alpha\beta +\gamma\right)^2}.
\end{equation}
Notice that if $\alpha\beta + \gamma \neq 0$ and neither constant $\alpha$, $\beta$
and $\gamma$ are trivial, then Ricci tensor can acts as a metric tensor. In that 
particular case, the spatial part of the ricci tensor plays an analogue role to 
the scale factor in the FLRW universe, thus
\begin{equation}
    \Sigma\left(\alpha,\beta,\gamma\right)t^{2\left(\frac{-\gamma}{\alpha\beta + \gamma}\right)} \longleftrightarrow a^2(t).
\end{equation}
Therefore, we can use the Ricci tensor to define the notion of distance, as 
long as is well behaved (nondegenerate). For this metric, unlike the one in FLRW 
universe, we have a singularity in the temporal part of the Ricci tensor at $t = 0$, 
meaning that the associated lapse function in this affine geometry is not well 
defined at that particular time. Additionally, the metriciy condition is not satisfied, 
meaning $\nabla_\lambda \mathcal{R}_{\mu\nu}$ does  not vanishes, and as a consequence 
of this, there are non-metricity effects $\mathcal{Q}^{\mathcal{R}}_{\mu\nu\lambda}$.

We can also compute Poplawski's metric like tensor which 
was defined in Eq. \eqref{Pop_tensor} using Eqs. \eqref{sol_p} and \eqref{sol_n}, leading to
\begin{align}
    \mathcal{P}_{tt} & = 0 & \mathcal{P}_{rr} & = \alpha^2\beta^2\left(t\left(\alpha\beta + \gamma\right)\right)^{-\frac{2\gamma}{\alpha\beta + \gamma}},
\end{align}
however, unlike the Ricci tensor, the Poplawski metric like tensor, is a 
degenerate tensor which can not be inverted, and therefore, it can not
act as a metric tensor.

\subsection{Matter interpretation from torsion fields}

The cosmological solutions allow us to have a well defined Ricci tensor, whose
spatial part was defined in Eq. \eqref{R_spatial}, this is a power-law solution,
where the coupling constant $\alpha$, $\beta$ and $\gamma$ are due to the presence 
of the torsion field. This type of solution can not be found in the torsion-free sector.

In order to provide a physical meaning we inquire the value of the parameters $\alpha$, 
$\beta$ and $\gamma$, that possess a well-known behavior in General Relativity, 
specifically in the FLRW universe. In the standard model of cosmology, there are
three well-known epochs: the radiation era, the non-relativistic matter era and the
accelerated expansion, each with its respective scale factor. 

We can find relations between the constants $\alpha$, $\beta$ and $\gamma$ in order
to have an affine scale factor $a_f(t)$ that can have the same behavior as the one from
FLRW Universe $a(t)$ with the big contrast that: in the standard model of cosmology,
the source to get a dynamical factor is the energy-momentum tensor, whereas in the
affine geometry, the \textit{source} is the torsion field, specifically its trace-less
part.

With the relation $\gamma = -\frac{\alpha\beta}{3}$, the affine scale factor turns to
\begin{equation}
    a_f(t) = a_0 t^{1/2},
\end{equation}
where $a_0$ is defined as 
\begin{equation}
    a_0 = \alpha^2 \hat{\beta},
\end{equation}
with $\hat{\beta} = -\beta^3$, where $\beta$ must be a negative number in order to have 
a real affine scale factor, which translate into an inequality for the coupling constants
\begin{equation}
    \label{constraint}
    3B_3 < 2B_4.
\end{equation}
This type of solution allow us to emulate the radiation era. 

In the same spirit, considering the case $\gamma = -\frac{2\alpha\beta}{5}$, leads to
\begin{equation}
    a_f(t) = a_{0}t^{2/3},
\end{equation}
where we defiend $a_0$ as 
\begin{equation}
    a_0 = \sqrt{-\alpha^{13/3}\beta^{10/3}}\left(\frac{2^{3/2}3^{2/3}}{5^{7/6}}\right),
\end{equation}    
which, in order to be real requires that $\beta <0$, which is consistent with constraint 
written in Eq. \eqref{constraint}. This type of solution allow us to emulate a non-relativistic 
matter era.

If we want to have an accelerated expansion due to torsion effects, we need to consider
\begin{equation}
    \label{acceleration_constraint}
    \ddot{a}_f > 0 \to \frac{\gamma\left(\alpha\beta + 2\gamma\right)}{\left(\alpha\beta + \gamma\right)^2}t^{\frac{\alpha\beta}{\alpha\beta + \gamma} - 3} > 0.
\end{equation}
the above condition, ensures a positive acceleration. In order to have a steady growth acceleration, it is required that 
\begin{equation}
    \label{restriction_acceleration}
    \alpha\beta < -\frac{3}{2}\gamma,
\end{equation}
this conditions ensures that the time evolution will be defined positive. Additionally, we have two branches coming from 
the numerator and denominator of Eq. \eqref{acceleration_constraint}
\begin{align}
    B1: \alpha\beta & > -2\gamma & \gamma & > 0, \label{B1_a} \\
    B2: \alpha\beta & < -2\gamma & \gamma & < 0. \label{B2_a}
\end{align}
Using Eq. \eqref{restriction_acceleration} along with Eqs. \eqref{B1_a} and \eqref{B2_a} allow us to define to intervals for the
constant $\alpha\beta$, for the first branch $B1$, we have $\alpha\beta \in \left[-2\gamma, -\frac{3}{2}\gamma\right]$, 
whereas for the second branch $B2$ we get $\alpha\beta <-2\gamma$. 

\section{Final remarks}
\label{sec:final_remarks}

In this work, we have analyzed some cosmological scenarios appearing in the context of the 
Polynomial Affine Gravity model with torsion effects, generalizing those considered in 
Refs.\cite{castillofelisola2019cosmological,Castillo_Felisola_2020}. Considering a non 
vanishing torsion tensor, yields an overdetermined system of differential equations, 
see Eqs. \eqref{Feq_1} to \eqref{Feq_5}, unlike the case without torsion effects, whose
field equations are reduced to the \textit{harmonic curvature} $\nabla_\alpha 
\mathcal{R}_{\beta\gamma}{}^{\alpha}{}_{\delta} = 0$, which through Bianchi's identity 
can be written as an anti symmetrized covariant derivative of the Ricci tensor $\nabla_{[\mu} 
\mathcal{R}_{\alpha]\beta} = 0$. For this particular case, the Ricci tensor is a
Codazzi tensor.

Without torsion effects, the field equation leads to one differential equation for two 
unknown functions (see eq.\eqref{Feq_5} with $\eta(t) = 0$), thus, the system can be 
solved parametrically, for example introducing an ansatz for $h(t)$ or $g(t)$. A priori, there 
is no physical reason to fix one of the functions, however, one could explore the subspace of 
this torsion-free field equation, by studying  integrability conditions such as ${R}_{\beta\gamma} = 0$ 
and $\nabla_\alpha \mathcal{R}_{\beta\gamma} = 0$. The former presents a system of two differential
equations for two unknown functions, which can be solved analytically. However, because the field
equations requires a vanishing Ricci tensor, in this subspace of solutions it is not possible
to find emergent metric tensor. 

The latter, yields a three differential equations, which can 
be solved analytically and provides with an exact solution for $h(t)$ and $g(t)$ functions. 
Even more, the Ricci tensor is well behaved and it can be used as a metric tensor. This case 
yields well-known de-Sitter/Anti de-Sitter space of solutions, where the cosmological constant is an
integration constant, see Ref.\cite{Castillo_Felisola_2020}.

The introduction of torsion fields with the tensors $\mathcal{B}$ and $\mathcal{A}$, induce 
nontrivial effects in the dynamics of the system. Due to the presence of the torsion tensor, 
there are two (possible) emergent descendant metric structures coming from the irreducible fields of the 
affine connection in the space of solution. Specifically, one comes from the symmetric part 
of the connection, which allow us to define the covariant derivative $\nabla^{\Gamma}$, and, 
the other one comes from  the anti-symmetric part of the affine connection, defined by the 
contraction of the product of two torsion tensors. 

Keeping in mind in the standard FLRW model of the universe, there are two fundamental objects, 
the metric tensor $g_{\mu\nu}$, and the energy-momentum tensor $T_{\mu\nu}$, we could argue
that in Polynomial Affine Gravity, the Ricci tensor $\mathcal{R}_{\mu\nu}$ can act as $g_{\mu\nu}$,
whereas $\mathcal{P}_{\mu\nu}$ emulate a $T_{\mu\nu}$. In that sense, the Ricci tensor provide us
an affine descendent scale factor, while the Poplawski like metric tensor emulate different types 
of matter effects.

Interestingly, the system of field equations eqs. \eqref{Feq_1} to \eqref{Feq_5} present four different 
branches of solution, of which only two branches can be solved exactly (non parametrically). The first
branch can be solved analytically defining completely the symmetric and skew symmetric part of the
affine connection. The solutions allow us, to define the Ricci tensor as an emergent metric
tensor in the space of solutions, however, the Poplawski tensor is a degenerate tensor, because its
temporal component its trivial, and therefore, it can not be interpreted as a metric tensor.

From the solution, it is possible to find relations for the constants $\alpha$, $\beta$ and $\gamma$
that allow us to define an affine scale factor that can emulate the standard scale factor coming
from the $\Lambda$CDM model, providing us with the three classical eras: radiation era, non-relativistic
matter era and the accelerated expansion. However, unlike FRLW Universe where the source of the scale
factor dynamics comes from the energy-momentum tensor, in the affine geometry, the \textit{source}
comes from  having non trivial torsion fields.

Moreover, the second branch also can be solved analytically for the special case where $\kappa \neq 0$. 
For this particular case, $g(t)$ is trivial and the symmetric part of the affine connection is
determined completely by the $h(t)$ function. As a consequence of this solution, the Ricci tensor
can be interpreted as a metric tensor, however, the affine scale factor coming from its spatial 
component is constant and provides no dynamics. Following a similar analysis, the $psi(t)$ it is
a constant defined by the geometrical factor $\kappa$ and the coupling constant, and
$\eta(t)$ has a non trivial inverse time dependence, defining completely the skew-symmetric part
of the affine connection. The Poplawski metric like tensor $\mathcal{P}_{\mu\nu}$ it is well defined,
however, the affine scale factor coming from its spatial it is also a constant (just like the Ricci
tensor), and therefore, it has no dynamics.


It has been shown that, the introduction of the torsion field in this affine model of gravity
allow us to obtain an affine scale factor coming from the symmetric part of the affine connection
in the space of solutions. Moreover, due to torsion effects, it is possible to recover the 
three classical epochs of the standard model of cosmology, whose \textit{source} in this affine gravity 
is the torsion tensor instead of an energy-momentum tensor.


\section{Acknowledgments}

We are specially thankful to the developers and maintainers of SageMath \cite{sagemath}, SageManifolds 
\cite{Gourgoulhon_2015,Gourgoulhon_2018}, and Cadabra \cite{peeters2018introducing,Peeters2018,Peeters_2007}. Those softwares were 
used extensively in our calculations.

\bibliographystyle{unsrt}
\bibliography{References}



\end{document}


